%% bare_jrnl.tex
%% V1.4b
%% 2015/08/26
%% by Michael Shell
%% see http://www.michaelshell.org/
%% for current contact information.
%%
%% This is a skeleton file demonstrating the use of IEEEtran.cls
%% (requires IEEEtran.cls version 1.8b or later) with an IEEE
%% journal paper.
%%
%% Support sites:
%% http://www.michaelshell.org/tex/ieeetran/
%% http://www.ctan.org/pkg/ieeetran
%% and
%% http://www.ieee.org/

%%*************************************************************************
%% Legal Notice:
%% This code is offered as-is without any warranty either expressed or
%% implied; without even the implied warranty of MERCHANTABILITY or
%% FITNESS FOR A PARTICULAR PURPOSE! 
%% User assumes all risk.
%% In no event shall the IEEE or any contributor to this code be liable for
%% any damages or losses, including, but not limited to, incidental,
%% consequential, or any other damages, resulting from the use or misuse
%% of any information contained here.
%%
%% All comments are the opinions of their respective authors and are not
%% necessarily endorsed by the IEEE.
%%
%% This work is distributed under the LaTeX Project Public License (LPPL)
%% ( http://www.latex-project.org/ ) version 1.3, and may be freely used,
%% distributed and modified. A copy of the LPPL, version 1.3, is included
%% in the base LaTeX documentation of all distributions of LaTeX released
%% 2003/12/01 or later.
%% Retain all contribution notices and credits.
%% ** Modified files should be clearly indicated as such, including  **
%% ** renaming them and changing author support contact information. **
%%*************************************************************************


% *** Authors should verify (and, if needed, correct) their LaTeX system  ***
% *** with the testflow diagnostic prior to trusting their LaTeX platform ***
% *** with production work. The IEEE's font choices and paper sizes can   ***
% *** trigger bugs that do not appear when using other class files.       ***                          ***
% The testflow support page is at:
% http://www.michaelshell.org/tex/testflow/



\documentclass[journal]{IEEEtran}
%
% If IEEEtran.cls has not been installed into the LaTeX system files,
% manually specify the path to it like:
% \documentclass[journal]{../sty/IEEEtran}

%**** LANGUAGE PACKAGES *****
\usepackage[spanish, shorthands=off]{babel}



% Some very useful LaTeX packages include:
% (uncomment the ones you want to load)


% *** MISC UTILITY PACKAGES ***
%
%\usepackage{ifpdf}
% Heiko Oberdiek's ifpdf.sty is very useful if you need conditional
% compilation based on whether the output is pdf or dvi.
% usage:
% \ifpdf
%   % pdf code
% \else
%   % dvi code
% \fi
% The latest version of ifpdf.sty can be obtained from:
% http://www.ctan.org/pkg/ifpdf
% Also, note that IEEEtran.cls V1.7 and later provides a builtin
% \ifCLASSINFOpdf conditional that works the same way.
% When switching from latex to pdflatex and vice-versa, the compiler may
% have to be run twice to clear warning/error messages.



% *** CITATION PACKAGES ***
%
\usepackage{cite}
% cite.sty was written by Donald Arseneau
% V1.6 and later of IEEEtran pre-defines the format of the cite.sty package
% \cite{} output to follow that of the IEEE. Loading the cite package will
% result in citation numbers being automatically sorted and properly
% "compressed/ranged". e.g., [1], [9], [2], [7], [5], [6] without using
% cite.sty will become [1], [2], [5]--[7], [9] using cite.sty. cite.sty's
% \cite will automatically add leading space, if needed. Use cite.sty's
% noadjust option (cite.sty V3.8 and later) if you want to turn this off
% such as if a citation ever needs to be enclosed in parenthesis.
% cite.sty is already installed on most LaTeX systems. Be sure and use
% version 5.0 (2009-03-20) and later if using hyperref.sty.
% The latest version can be obtained at:
% http://www.ctan.org/pkg/cite
% The documentation is contained in the cite.sty file itself.


% *** GRAPHICS RELATED PACKAGES ***
%
\ifCLASSINFOpdf
 \usepackage[pdftex]{graphicx}
 \usepackage{subfigure} % subfiguras
  % declare the path(s) where your graphic files are 
 \graphicspath{ {images/} }
  % and their extensions so you won't have to specify these with
  % every instance of \includegraphics
  \DeclareGraphicsExtensions{.pdf,.jpeg,.png}
\else
  % or other class option (dvipsone, dvipdf, if not using dvips). graphicx
  % will default to the driver specified in the system graphics.cfg if no
  % driver is specified.
  % \usepackage[dvips]{graphicx}
  % declare the path(s) where your graphic files are
  % \graphicspath{{../eps/}}
  % and their extensions so you won't have to specify these with
  % every instance of \includegraphics
  % \DeclareGraphicsExtensions{.eps}
\fi
% graphicx was written by David Carlisle and Sebastian Rahtz. It is
% required if you want graphics, photos, etc. graphicx.sty is already
% installed on most LaTeX systems. The latest version and documentation
% can be obtained at: 
% http://www.ctan.org/pkg/graphicx
% Another good source of documentation is "Using Imported Graphics in
% LaTeX2e" by Keith Reckdahl which can be found at:
% http://www.ctan.org/pkg/epslatex
%
% latex, and pdflatex in dvi mode, support graphics in encapsulated
% postscript (.eps) format. pdflatex in pdf mode supports graphics
% in .pdf, .jpeg, .png and .mps (metapost) formats. Users should ensure
% that all non-photo figures use a vector format (.eps, .pdf, .mps) and
% not a bitmapped formats (.jpeg, .png). The IEEE frowns on bitmapped formats
% which can result in "jaggedy"/blurry rendering of lines and letters as
% well as large increases in file sizes.
%
% You can find documentation about the pdfTeX application at:
% http://www.tug.org/applications/pdftex





% *** MATH PACKAGES ***
%
\usepackage{amsmath}
% A popular package from the American Mathematical Society that provides
% many useful and powerful commands for dealing with mathematics.
%
% Note that the amsmath package sets \interdisplaylinepenalty to 10000
% thus preventing page breaks from occurring within multiline equations. Use:
%\interdisplaylinepenalty=2500
% after loading amsmath to restore such page breaks as IEEEtran.cls normally
% does. amsmath.sty is already installed on most LaTeX systems. The latest
% version and documentation can be obtained at:
% http://www.ctan.org/pkg/amsmath





% *** SPECIALIZED LIST PACKAGES ***
%
\usepackage{algorithmic}
% algorithmic.sty was written by Peter Williams and Rogerio Brito.
% This package provides an algorithmic environment fo describing algorithms.
% You can use the algorithmic environment in-text or within a figure
% environment to provide for a floating algorithm. Do NOT use the algorithm
% floating environment provided by algorithm.sty (by the same authors) or
% algorithm2e.sty (by Christophe Fiorio) as the IEEE does not use dedicated
% algorithm float types and packages that provide these will not provide
% correct IEEE style captions. The latest version and documentation of
% algorithmic.sty can be obtained at:
% http://www.ctan.org/pkg/algorithms
% Also of interest may be the (relatively newer and more customizable)
% algorithmicx.sty package by Szasz Janos:
% http://www.ctan.org/pkg/algorithmicx



\usepackage{xspace} 

% *** ALIGNMENT PACKAGES ***
%
\usepackage{array}
% Frank Mittelbach's and David Carlisle's array.sty patches and improves
% the standard LaTeX2e array and tabular environments to provide better
% appearance and additional user controls. As the default LaTeX2e table
% generation code is lacking to the point of almost being broken with
% respect to the quality of the end results, all users are strongly
% advised to use an enhanced (at the very least that provided by array.sty)
% set of table tools. array.sty is already installed on most systems. The
% latest version and documentation can be obtained at:
% http://www.ctan.org/pkg/array


% IEEEtran contains the IEEEeqnarray family of commands that can be used to
% generate multiline equations as well as matrices, tables, etc., of high
% quality.




% *** SUBFIGURE PACKAGES ***
%\ifCLASSOPTIONcompsoc
%  \usepackage[caption=false,font=normalsize,labelfont=sf,textfont=sf]{subfig}
%\else
%  \usepackage[caption=false,font=footnotesize]{subfig}
%\fi
% subfig.sty, written by Steven Douglas Cochran, is the modern replacement
% for subfigure.sty, the latter of which is no longer maintained and is
% incompatible with some LaTeX packages including fixltx2e. However,
% subfig.sty requires and automatically loads Axel Sommerfeldt's caption.sty
% which will override IEEEtran.cls' handling of captions and this will result
% in non-IEEE style figure/table captions. To prevent this problem, be sure
% and invoke subfig.sty's "caption=false" package option (available since
% subfig.sty version 1.3, 2005/06/28) as this is will preserve IEEEtran.cls
% handling of captions.
% Note that the Computer Society format requires a larger sans serif font
% than the serif footnote size font used in traditional IEEE formatting
% and thus the need to invoke different subfig.sty package options depending
% on whether compsoc mode has been enabled.
%
% The latest version and documentation of subfig.sty can be obtained at:
% http://www.ctan.org/pkg/subfig




% *** FLOAT PACKAGES ***
%
%\usepackage{fixltx2e}
% fixltx2e, the successor to the earlier fix2col.sty, was written by
% Frank Mittelbach and David Carlisle. This package corrects a few problems
% in the LaTeX2e kernel, the most notable of which is that in current
% LaTeX2e releases, the ordering of single and double column floats is not
% guaranteed to be preserved. Thus, an unpatched LaTeX2e can allow a
% single column figure to be placed prior to an earlier double column
% figure.
% Be aware that LaTeX2e kernels dated 2015 and later have fixltx2e.sty's
% corrections already built into the system in which case a warning will
% be issued if an attempt is made to load fixltx2e.sty as it is no longer
% needed.
% The latest version and documentation can be found at:
% http://www.ctan.org/pkg/fixltx2e


%\usepackage{stfloats}
% stfloats.sty was written by Sigitas Tolusis. This package gives LaTeX2e
% the ability to do double column floats at the bottom of the page as well
% as the top. (e.g., "\begin{figure*}[!b]" is not normally possible in
% LaTeX2e). It also provides a command:
%\fnbelowfloat
% to enable the placement of footnotes below bottom floats (the standard
% LaTeX2e kernel puts them above bottom floats). This is an invasive package
% which rewrites many portions of the LaTeX2e float routines. It may not work
% with other packages that modify the LaTeX2e float routines. The latest
% version and documentation can be obtained at:
% http://www.ctan.org/pkg/stfloats
% Do not use the stfloats baselinefloat ability as the IEEE does not allow
% \baselineskip to stretch. Authors submitting work to the IEEE should note
% that the IEEE rarely uses double column equations and that authors should try
% to avoid such use. Do not be tempted to use the cuted.sty or midfloat.sty
% packages (also by Sigitas Tolusis) as the IEEE does not format its papers in
% such ways.
% Do not attempt to use stfloats with fixltx2e as they are incompatible.
% Instead, use Morten Hogholm'a dblfloatfix which combines the features
% of both fixltx2e and stfloats:
%
% \usepackage{dblfloatfix}
% The latest version can be found at:
% http://www.ctan.org/pkg/dblfloatfix




%\ifCLASSOPTIONcaptionsoff
%  \usepackage[nomarkers]{endfloat}
% \let\MYoriglatexcaption\caption
% \renewcommand{\caption}[2][\relax]{\MYoriglatexcaption[#2]{#2}}
%\fi
% endfloat.sty was written by James Darrell McCauley, Jeff Goldberg and 
% Axel Sommerfeldt. This package may be useful when used in conjunction with 
% IEEEtran.cls'  captionsoff option. Some IEEE journals/societies require that
% submissions have lists of figures/tables at the end of the paper and that
% figures/tables without any captions are placed on a page by themselves at
% the end of the document. If needed, the draftcls IEEEtran class option or
% \CLASSINPUTbaselinestretch interface can be used to increase the line
% spacing as well. Be sure and use the nomarkers option of endfloat to
% prevent endfloat from "marking" where the figures would have been placed
% in the text. The two hack lines of code above are a slight modification of
% that suggested by in the endfloat docs (section 8.4.1) to ensure that
% the full captions always appear in the list of figures/tables - even if
% the user used the short optional argument of \caption[]{}.
% IEEE papers do not typically make use of \caption[]'s optional argument,
% so this should not be an issue. A similar trick can be used to disable
% captions of packages such as subfig.sty that lack options to turn off
% the subcaptions:
% For subfig.sty:
% \let\MYorigsubfloat\subfloat
% \renewcommand{\subfloat}[2][\relax]{\MYorigsubfloat[]{#2}}
% However, the above trick will not work if both optional arguments of
% the \subfloat command are used. Furthermore, there needs to be a
% description of each subfigure *somewhere* and endfloat does not add
% subfigure captions to its list of figures. Thus, the best approach is to
% avoid the use of subfigure captions (many IEEE journals avoid them anyway)
% and instead reference/explain all the subfigures within the main caption.
% The latest version of endfloat.sty and its documentation can obtained at:
% http://www.ctan.org/pkg/endfloat
%
% The IEEEtran \ifCLASSOPTIONcaptionsoff conditional can also be used
% later in the document, say, to conditionally put the References on a 
% page by themselves.




% *** PDF, URL AND HYPERLINK PACKAGES ***
%
%\usepackage{url}
% url.sty was written by Donald Arseneau. It provides better support for
% handling and breaking URLs. url.sty is already installed on most LaTeX
% systems. The latest version and documentation can be obtained at:
% http://www.ctan.org/pkg/url
% Basically, \url{my_url_here}.


\usepackage{authblk}

% *** Do not adjust lengths that control margins, column widths, etc. ***
% *** Do not use packages that alter fonts (such as pslatex).         ***
% There should be no need to do such things with IEEEtran.cls V1.6 and later.
% (Unless specifically asked to do so by the journal or conference you plan
% to submit to, of course. )


% correct bad hyphenation here
\hyphenation{op-tical net-works semi-conduc-tor}

% Paquete para saltos del inea
\usepackage[utf8]{inputenc}


% Paquete para utilizar colores
\usepackage{color}



\begin{document}
%
% paper title
% Titles are generally capitalized except for words such as a, an, and, as,
% at, but, by, for, in, nor, of, on, or, the, to and up, which are usually
% not capitalized unless they are the first or last word of the title.
% Linebreaks \\ can be used within to get better formatting as desired.
% Do not put math or special symbols in the title.
\title{ 10 Experimentos Científicos más Bellos de la Historia }
%
%
% author names and IEEE memberships
% note positions of commas and nonbreaking spaces ( ~ ) LaTeX will not break
% a structure at a ~ so this keeps an author's name from being broken across
% two lines.
% use \thanks{} to gain access to the first footnote area
% a separate \thanks must be used for each paragraph as LaTeX2e's \thanks
% was not built to handle multiple paragraphs
%

\author{Manuel Figueroa,\IEEEmembership{ Estudiante, ITCR,}
        Esteban Leandro,\IEEEmembership{ Estudiante, ITCR}}% <-this % stops a space
\affil[]{\textit{MC-7201 Introducción a la Investigación}}
\affil[]{\textit{Instituto Tecnológico de Costa Rica}}
\affil[]{\textit{\{mfigueroacr, elc790\}@gmail.com}}

% note the % following the last \IEEEmembership and also \thanks - 
% these prevent an unwanted space from occurring between the last author name
% and the end of the author line. i.e., if you had this:
% 
% \author{....lastname \thanks{...} \thanks{...} }
%                     ^------------^------------^----Do not want these spaces!
%
% a space would be appended to the last name and could cause every name on that
% line to be shifted left slightly. This is one of those "LaTeX things". For
% instance, "\textbf{A} \textbf{B}" will typeset as "A B" not "AB". To get
% "AB" then you have to do: "\textbf{A}\textbf{B}"
% \thanks is no different in this regard, so shield the last } of each \thanks
% that ends a line with a % and do not let a space in before the next \thanks.
% Spaces after \IEEEmembership other than the last one are OK (and needed) as
% you are supposed to have spaces between the names. For what it is worth,
% this is a minor point as most people would not even notice if the said evil
% space somehow managed to creep in.



% The paper headers
\markboth{ININ Tarea Corta 2: Experimentos más bellos de la historia , Octubre 2019}%
{Shell \MakeLowercase{\textit{et al.}}: Experimentos más bellos de la historia}
% The only time the second header will appear is for the odd numbered pages
% after the title page when using the twoside option.
% 
% *** Note that you probably will NOT want to include the author's ***
% *** name in the headers of peer review papers.                   ***
% You can use \ifCLASSOPTIONpeerreview for conditional compilation here if
% you desire.




% If you want to put a publisher's ID mark on the page you can do it like
% this:
%\IEEEpubid{0000--0000/00\$00.00~\copyright~2015 IEEE}
% Remember, if you use this you must call \IEEEpubidadjcol in the second
% column for its text to clear the IEEEpubid mark.



% use for special paper notices
%\IEEEspecialpapernotice{(Invited Paper)}




% make the title area
\maketitle

% As a general rule, do not put math, special symbols or citations
% in the abstract or keywords.
% \begin{abstract}

% \end{abstract}

% Note that keywords are not normally used for peerreview papers.
\begin{IEEEkeywords}
\LaTeX\xspace , Introducción a la investigación, Tarea Corta, Experimentos, Historia.
\end{IEEEkeywords}






% For peer review papers, you can put extra information on the cover
% page as needed:
% \ifCLASSOPTIONpeerreview
% \begin{center} \bfseries EDICS Category: 3-BBND \end{center}
% \fi
%
% For peerreview papers, this IEEEtran command inserts a page break and
% creates the second title. It will be ignored for other modes.
\IEEEpeerreviewmaketitle



\section{Eratóstenes y la circunferencia de la tierra}
\begin{center}
  \begin{figure}[h!]
  \includegraphics[width=55mm]{eratostenes1.jpeg}
  \caption{Eratóstenes. Una pintura de Bernardo Strozzi \emph{Tomado de Google Imágenes}}
  \end{figure}
\end{center}
%*******************************Usos acádemicos extension e importancia*******************************%
\subsection{Contexto Histórico}
Eratóstenes fue un académico de la antigua Grecia (276 a.C - 195 a.C) conocido por realizar la primera medición conocida
de la Tierra. Eratóstenes parte de la suposición griega de que la tierra es esférica,
y que en comparación con otros cuerpos celestes,
esta era diminuta. Esto se explica en la obra \emph{Acerca del clelo},
de Aristóteles y escrita un siglo antes de Eratóstenes.
Entre los argumentos lógicos de la obra se mencionan entre otros hechos que los viajeros ven 
estrellas distintas si viajan al norte o al sur y que algunas estrellas visibles en lugares como Egipto o Chipre no son visibles en lugares más 
septentrionales.
Eratóstenes nació al norte de África, y se educó en Atenas, fue un pensador influyente en muchas áreas y escribió \emph{Geographica}, una obra de 
geografía conocida por ser la primera en utilizar el sistema de parelos y meridianos conocido en la actualidad.

\subsection{El experimento}
Eratóstenes buscaba obtener una medición más precisa y verificar o desmentir estimaciones anteriores del tamaño real de la Tierra.
Aristóteles calculaba este tamaño en 400.000 estadios que es aproximadamente unos 64.000 kilómetros, algo lejos del valor real del diámetro de la Tierra (40.000 Km)
Eratóstenes asumió que si la tierra era de hecho un cuerpo pequeño y esférico, entonces otros cuerpos como el Sol deberían de encontrarse muy lejos  de manera que sus rayos
deberían ser prácticamente paralelos en todos los puntos de la Tierra.

Utilizando este hecho como base de su experimento y 
conociendo por relatos que en la ciudad de Siena (Asuán, Egipto)
durante el solsticio de verano el sol del mediodía se ubicaba justo por encima de la cabeza. De este modo
no se proyectaba ninguna sombra en un objeto vertical.

Al mismo tiempo en Alejandría, ciudad ubicada al norte de Siena, se conocía que nunca se podía observar al sol 
directamente sobre la cabeza, razón por la cuál los objetos verticales siempre proyectaban una sombra.

Este hecho, sirvió a Eratóstenes para realizar los cálculos de medición de la circunferencia de la tierra con gran precisión.
La simplicidad del experimento permite determinar dimensiones cósmicas midiendo unicamente la longitud de la sombra proyctada por 
un reloj solar en Alejandría, mientras que en Siena ocurría el solsticio y no se proyectaba sombra.

De manera similar al siguiente gráfico:
\begin{center}
  \begin{figure}[h!]
  \includegraphics[width=60mm]{Eratostenes.jpg}
  \caption{Cálculo realizado por Eratóstenes. \emph{Tomado de Google Imágenes}}
  \end{figure}
\end{center}

Deacuerdo a la geometría Euclideana, los ángulos interiores
de una línea que interseca dos líneas paralelas son iguales, por lo tanto el ángulo formado por el Zenith y el Sol, es igual al formado
por los radios desde el centro de la tierra a Siena y Alejandría.

Esto le sirvió para determinar la fracción de la circunferencia representada por la distancia ya conocida entre Siena y Alejandría que había sido 
determinada por los topográfos reales del gobierno Egipcio, con esto logró determinar el tamaño de la circunferencia 
de la Tierra en unos 252 000 estadios, lo que es aproximadamente 40.200 Km una cifra bastante cercana a la aceptada en la actualidad de 40.075Km \cite{crease_2014}

\section{Hershey - Chase: Función genética del ADN}
\begin{center}
  \begin{figure}[h!]
  \includegraphics[width=55mm]{hersheyChase.jpg}
  \caption{Alfred Hershey y Martha Chase \emph{Tomado de Google Imágenes}}
  \end{figure}
\end{center}
\subsection{Contexto Histórico}

A principios del siglo XX se aceptaba que el material genético de las células
era formado por proteínas. Esto principalmente a que se conocía que la estructura 
del ADN, por las investigaciones de Phoebus Levene en 1933, consistía de cuatro elementos llamados
nucleótidos.
Debido a esta limitación en la cantidad de bloques que formaban las estructuras de ADN se consideraba
imposible que este sirviese como mecánismo para transferir información genética, como el color de piel, ojos, entre otros.
Las proteínas, elementos también presentes en las células ofrecían un mayor factor de diversidad y podían combinarse de muchas más maneras. Por esta
razón se creía que eran estas las encargadas de transmitir las características en cada generación.

En 1935, Oswald Avery realizó una serie de experimentos que mostraron que el ADN facilitaba un fénomeno genético en las bacterias, pudiendo demostrar que el factor de herencia 
que causaba transformaciones en las bacterias contenía ADN. Sin embargo, no se pudo descartar que otros componentes sin ADN estuviesen involucrados en dicha transformación.
Por esta razón, muchos científicos seguían considerando a las proteínas como las encargadas de transmitir la herencia genética de las células.

\subsection{El experimento}
En 1951, los científicos Alfred Hershey y Martha Chase iniciaron una serie de experimentos con el objetivo de desacreditar
las afirmaciones de Avery. En sus experimentos se analizó como los bacteriófagos infectaban las bacterias.
Descubrieron que cuando un fago infecta a una bacteria, inicialmente se pega al exterior de la bacteria y despues inserta parte de su contenido 
al interior de la bacteria, lo que le permite replicarse dentro de la misma y generar nuevos bacteriófagos que invadan a las células cercanas.

La técnica utilizada por Hershey y Chase consistía en usar etiquetas de isótopos radiactivos.
Los elementos químicos pueden existir en diferentes formas estructurales denominadas isótopos, que pueden 
tener diferentes niveles de radiactividad que pueden ser detectados por los científicos y de esta manera determinar si las partes etiquetadas fueron transmitidas de los fagos a las bacterias.

Etiquetando la parte de proteínas  del bacteriófago con isótopos de azufre y el ADN con fósforo radiactivo, y utilizando una licuadora común, descubrieron que las proteínas permitían al fago
pegarse a la membrana superficial de la bactería y lo que se inyectaba dentro del interior de la bactería era de hecho el ADN del bacteriófago, y por lo tanto lo
que permitía la replicación de nuevos bacteriófagos en el interior de la bacteria infectada.

Los resultados de medir la mezcla descubrieron que al licuar las bacterias infectadas se removía hasta el 80\% de las proteínas marcas y solamente cerca del 40\% del ADN marcado
indicando que el material restante se había incorporado al interior de las céulas.

Con este experimento demostraron que Avery estaba en lo correcto y que el componente de la herencia genética
es en realidad el ADN y no las proteínas como se creía.

Por esta serie de experimentos Hershey recibe el premio Nobel en 1969.\cite{hernandez_2019}
\begin{center}
  \begin{figure}[h!]
  \includegraphics[width=60mm]{hershey.jpg}
  \caption{Experimento Hershey-Chase. \emph{Tomado de Google Imágenes}}
  \end{figure}
\end{center}

\section{Luigi Galvani: Electricidad Animal}

\begin{center}
  \begin{figure}[h!]
  \includegraphics[width=55mm]{Luigi_Galvani.jpg}
  \caption{Luigi Galvani. \emph{Tomado de Google Imágenes}}
  \end{figure}
\end{center}

\subsection{Contexto Histórico}
A mediados del siglo XVIII, la electricidad era un tema importante y que acaparaba la atención 
de muchos científicos de la época, debatiendo si la electricidad era un vapor, un fluido o como Benjamín
Franklin especulaba una serie de partículas.

En Abril de 1786, Luigi Galvani un profesor de anatomía, estuvo experimentando con la estimulación de los nervios
de las ranas mediante el uso de electricidad usando un generador o aplicando descargas desde una botella de Leyden.

\subsection{El experimento}
Luigi Galvani creía que los movimientos musculares eran causados por una electricidad natural producida 
por los seres vivos y que aplicar electricidad de origen artificial tenía el mismo efecto.

Galvani tomó varias ranas y verificó que al cerrar un circuito conectando el nervio con el músculo utilizando algún conductor 
se producía una reacción que hacía que las patas de las ranas se movieran y sin ninguna fuente externa de energía 
era fácil pensar que la electricidad estaba almacenada en el interior del animal.

En 1971, publicó sus hallazgos en \emph{Commentary on the Effect of Electricity on Muscular Motion}, donde efectivamente Galvani proponía que 
los músculos de las ranas funcionaban como una botella de Lynden, almanecenando y liberando algún tipo de electricidad orgánica.

Uno de los grandes detractores de esta teoría fue Alessandro Volta, físico y químico Italiano. Volta aseguraba que el efecto de movimiento en 
los músculos era provocado por la electricidad provocada al utilizar dos metales conductores, lo que denominó electricidad bimetálica.

Galvani refutó estas afirmaciones demostrando que el mismo efecto podría lograrse usando el mismo metal o incluso carbón para realizar la conexión.
Volta asumió que esto solamente era efecto de las impurezas del metal que anque fuesen del mismo tipo igual generan alguna 
cantidad de electricidad bimetálica.

Finalmente Galvani optó por eliminar los metales y conductores de sus experimentos simplemente haciendolo con sus manos y provocando el mismo efecto, incluso haciendo el contacto directamente entre el nervio 
y el músculo sin ningún conductor involucrado, probando así que la electricidad no provenía de afuera si no dentro del animal mismo.

Volta y Galvani estaban en lo correcto, la electricidad puede ser producida por la reacción de distintos conductores inventando así las baterías. De igual manera
los experimentos de Galvani probaron que en efecto el movimiento muscular es provocado por una reacción electroquímica y no por una éterea fuerza vital.\cite{johnson_2014}

\begin{center}
  \begin{figure}[h!]
  \includegraphics[width=55mm]{frogs.png}
  \caption{Experimentos de Galvani. \emph{Tomado de Google Imágenes}}
  \end{figure}
\end{center}
%\section{Conclusion}
%The conclusion goes here.

\section{Henry Canvendish: Densidad de la tierra}

\begin{center}
  \begin{figure}[h!]
  \includegraphics[width=55mm]{Cavendish.jpg}
  \caption{Henry Cavendish. \emph{Tomado de Google Imágenes}}
  \end{figure}
\end{center}

\subsection{Contexto Histórico}
Henry Cavendish (1731-1810) fue un destacado físico y químico británico,
 nacido en el Reino de Cerdeña en Francia, y uno de los mayores científicos de la historia.
 Cavendish tuvo la suerte de obtener una gran herencia que le sirvió para financiar sus experimentos
 y se le reconoce también el haber determinado la composición química del agua.
 
 \subsection{El experimento}
Uno de los mas famosos experimentos de Cavendish es el cálculo de la densidad terrestre, extraído directamente
del interés de Cavendish en la ciencia Newtoniana.
Sir Issac Newton había propuesto que todos los objetos en la Tierra son atraídos por la fuerza
de gravedad. Esta fuerzza es proporcional al tamaño de la masa de los objetos y a la distancia entre ellos. Basícamente, esto implicaba 
que entre mayor la masa de un objeto más fuerte la fuerza de atracción de este.
Sin embargo, Newton dejó como incognitas la constante gravitacional y la masa terrestre. El encontrar cualquiera de estos valores permitía
descubrir el otro mediante cálculos simples.

Como se conocía que la constante gravitacional es la misma para todos los objetos, el método más lógico de resolver
la incognita es midiendo la fuerza de atracción gravitacional entre dos objetos de masa conocida. Suena sencillo en teoría pero en la práctica
los objetos de masa conocida son muy pequeños para que exista una fuerza de attraccion gravitacional medible.

En la década de 1798 mediante el experimento de Cavendish, utilizando una balanza de torsión, se obtuvo la primera
medida de la constante gravitacional de la Tierra, y consecuentemente la densidad del planeta Tierra con una precisión asombrosa para la época.
El aparato construido por Cavendish, y colocado en un ambiente que eliminaba variables externas como el viento y la temperatura
podía calcular la fuerza de atracción ejercida sobre dos bolas de plomo de igual masa, con un gran grado de exactitud.

Utilizando este equipo Cavendish determinó que la densidad promedio
terrestre era de aproximadamente 5,5 veces la del agua y estimó 
la masa terrestre en $6,6 \times 10^{21}$
toneladas muy cercano al valor aceptado en la actualidad de
$5,97 \times 10^{24}$ kilogramos.\cite{shectman_2003}

\begin{center}
  \begin{figure}[h!]
  \includegraphics[width=55mm]{cavendish2.jpg}
  \caption{Balanza de torsión de Henry Canvendish. \emph{Tomado de Google Imágenes}}
  \end{figure}
\end{center}










\section{Thomas Young: La naturaleza ondulatoria de la luz}

\subsection{Contexto Histórico}
Thomas Young nació el 13 de junio de 1773 en Milverton, Inglaterra. Fue un niño prodigio, a los dos años ya leía y a los seis había leído dos veces la Biblia de principio a fin. Conocía una docena de lenguas incluidas el latín y el griego antiguo. Estudió Medicina, sin mucho éxito como médico, parte debido a su poca habilidad para reconfortar a los pacientes. Con veintiocho años abandonó la práctica médica para unirse a la Royal Institution de Londres. 

Fue uno de los primeros en descifrar jeroglíficos egipcios y desempeñó un papel esencial en la descodificación de la piedra de Rosetta. También es célebre por su experimento de la doble rendija que mostraba la naturaleza ondulatoria de la luz.

Young estudió la visión y el ojo humano, propuso la teoría tricromática de la visión confirmada ciento cincuenta años después. Investigó sobre el sonido, la audición y la voz humana y fue entonces cuando se preguntó si el sonido y la luz no tendrían la misma naturaleza ondulatoria.

\begin{center}
  \begin{figure}[h!]
  \includegraphics[width=50mm]{thomas_young.jpg}
  \caption{Thomas Young Davis. \emph{Tomado de Google Imágenes}}
  \end{figure}
\end{center}

\subsection{El experimento}

Su contribución fundamental al campo de la luz es el experimento de la doble rendija, considerado no sólo como uno de los experimentos más bellos de la física, sino también el experimento favorito con luz. Con este experimento Young desafió las teorías de Isaac Newton y demostró que la luz es una onda, que probaba que la luz sufre el fenómeno de las interferencias que es propio de las ondas. Entre 1801 y 1803 presentó una serie de conferencias en la Royal Society subrayando la teoría ondulatoria de la luz y añadiendo a la misma un nuevo concepto fundamental, el principio de interferencia. 

El experimento de la doble rendija es simple y permitió a Thomas Young demostrar de forma convincente y por primera vez la naturaleza ondulatoria de la luz. Cuando las ondas provenientes de dos rendijas estrechas se superponen sobre una pantalla colocada a cierta distancia paralela a la línea que conecta estas rendijas, aparece en la pantalla un patrón de franjas claras y oscuras espaciadas regularmente (patrón de interferencia).

Esta es la primera prueba clara de que luz más luz puede dar lugar a oscuridad. En la interferencia tiene lugar una redistribución espacial de la intensidad luminosa sin que se viole la conservación de la energía. 

Este fenómeno se conoce como interferencia y con este experimento se corroboraron las ideas intuitivas de Huygens respecto al carácter ondulatorio de la luz. Thomas Young esperaba este resultado pues creía firmemente en la teoría ondulatoria de la luz y su juicio éste había sido el más importante de sus muchos logros científicos.

\begin{center}
  \begin{figure}[h!]
  \includegraphics[width=70mm]{ondas_luz.jpg}
  \caption{Ondas de Luz. \emph{Tomado de Google Imágenes}}
  \end{figure}
\end{center}

El 12 de noviembre 1801 presentó ante la Royal Society la Bakerian Lecture titulada "On the Theory of Light and Colours" (Sobre la Teoría de la Luz y los Colores) y el 24 de noviembre de 1803 también la Bakerian Lecture "Experiments and Calculations relative to Physical Optics" (Experimentos y cálculos relativos a la óptica física).

En esta última presentaba la demostración experimental de la ley general de la interferencia de la luz y una inferencia argumentativa sobre la naturaleza de la luz, concluyendo que la luz era una onda. Como todas las ondas conocidas necesitaban un medio material para su propagación, como sucede con las ondas sonoras o las ondas en el agua.

Young consideró que la luz se propagaba en un medio, el éter luminífero, concluyendo que «A luminiferous Ether pervades de Universe, rare and elastic in high degree» (Un éter luminífero impregna todo el Universo, raro y elástico en alto grado) y afirmó de forma contundente que «Radiant light consists in Undulations of the luminiferous Ether» (la luz radiante consiste en ondulaciones del éter luminífero). 

Asimismo señaló que la sensación de los diferentes colores depende de la distinta frecuencia de las vibraciones de la luz que excita la retina.

\begin{center}
  \begin{figure}[h!]
  \includegraphics[width=70mm]{diagrama_interferencial.jpg}
  \caption{Diagrama interferencial observado por Young. \emph{Tomado de Google Imágenes}}
  \end{figure}
\end{center}

El experimento de las dos rendijas pone de manifiesto el proceso de interferencia óptica, nombre con que Thomas Young designó los procesos constructivos y destructivos de la composición de ondas y con el que también es conocido desde entonces.

En el año 1803 casi nadie aceptó de forma inmediata las ideas de Young sobre la naturaleza de la luz. Young publicó en 1807 su magnus opus, "A Course of Lectures on Natural Philosophy and the Mechanical Arts", consistente en dos volúmenes con más de mil quinientas páginas y que fue descrito por el físico Joseph Larmor (1857-1942) como el más grande y el más original de todos los cursos publicados.

Gracias a las contribuciones realizadas por Augustin Fresnel, la teoría ondulatoria de la luz –que Young demostró en su famoso experimento– fue finalmente aceptada.





\hfill \break
\section{Jean Léon Foucault: Péndulo de Foucault}

\subsection{Contexto Histórico}

Jean Léon Foucault nació en Francia el 18 de septiembre de 1819. Considerado como el fundador de la moderna técnica de construcción de los grandes telescopios, trabajó en la determinación de la velocidad de la luz. Obtuvo importantes distinciones y condecoraciones en el campo de la investigación científica. Fue un físico destacado, entre otras cosas, por su famoso péndulo.

Su experimento más famoso empezó en 1850, cuando observó que un péndulo permanecía oscilando en el mismo plano mientras se hacía rotar el aparato. Foucault usó entonces el péndulo para demostrar la rotación de la Tierra.


\begin{center}
  \begin{figure}[h!]
  \includegraphics[width=60mm]{leon_fourt.jpg}
  \caption{Jean Léon Foucault. \emph{Tomado de Google Imágenes}}
  \end{figure}
\end{center}

Un péndulo es algo que cuelga de un punto fijo, que cuando se suelta gira hacia abajo por la fuerza de gravedad, y luego hacia arriba por la inercia. La gravedad es una fuerza que atrae los objetos hacia el suelo, mientras que la inercia es la tendencia de un cuerpo en movimiento para continuar su movimiento a menos que actúe sobre él otra fuerza.

Los péndulos son muy útiles para la ciencia, ya que sirven no sólo para medir la rotación de la Tierra, sino también para medir la aceleración debido a la gravedad, algo importante para determinar la forma de la Tierra y la distribución de los materiales dentro de ella.

\subsection{El experimento}

Enunció una ecuación en la que se relacionaba el período de rotación del plano con la latitud de la Tierra, en una exhibición pública, suspendió una esfera de hierro de 28 kilogramos de un cable de acero de 67 metros, desde la cúpula del Panteón en París, cuyo comportamiento vino a corroborar sus cálculos.

Si se observa un péndulo, de pié y fijo a la Tierra, luego de este realizar varias oscilaciones se verá un pequeño desplazamiento del punto donde el péndulo alcanza su máxima apertura, se observará un pequeño giro del plano de oscilación del péndulo.

Si nuestro planeta Tierra estuviera inmóvil en el espacio, se podría observar que el plano de oscilación del péndulo no cambia. Esta es la argumentación que usó Foucault para mostrar que su experimento finalmente demostraba la rotación de la Tierra.

Considerando primero un péndulo oscilando justo en el polo Norte. Dado que el eje de rotación de la Tierra pasa por los polos, el piso rota en sentido antihorario, con respecto a las estrellas lejanas. Como el plano de oscilación del péndulo no cambia con respecto a esas estrellas, un observador fijo al suelo, verá el plano de oscilación del péndulo dar una vuelta completa en el sentido horario en ese mismo tiempo. 

Por su parte, el observador fijo a la Tierra, en el polo Sur, observará que el plano de oscilación del péndulo dará una vuelta completa en el mismo tiempo pero en sentido antihorario.

\begin{center}
  \begin{figure}[h!]
  \includegraphics[width=75mm]{pendulo_fourt.jpg}
  \caption{Péndulo de Foucault. \emph{Tomado de Google Imágenes}}
  \end{figure}
\end{center}

¿Qué sucede si se realiza el experimento justo en un punto de la línea del Ecuador?

A diferencia, de lo que ocurre en los polos, en el Ecuador el eje de rotación de la Tierra es paralelo al suelo y por lo tanto el suelo no gira con respecto al eje. De esta manera el observador fijo al suelo ve al péndulo oscilar siempre en el mismo plano. Usando un péndulo sobre la Línea del Ecuador no se puede detectar la rotación de la Tierra.

Ahora considerando un lugar entre el polo Sur y el Ecuador, por ejemplo alguna ubicación en Argentina. La base del péndulo, en el suelo, no es paralela ni perpendicular al eje de rotación de la Tierra, por lo tanto el suelo rotará, pero de manera más lenta que en los polos. Así, el observador fijo en la Tierra verá que el plano de oscilación del péndulo gira en sentido antihorario (porque está en el hemisferio Sur), pero tardará más tiempo en dar una vuelta completa.

La descripción matemática del movimiento del péndulo en cualquier latitud es un tanto compleja. Con ella se encuentra la fórmula que determina el tiempo (en horas) que tarda el péndulo en efectuar un giro completo. Esta es:

\begin{equation}
    T=24/sen(A),
\end{equation}

en que A es la latitud donde se encuentra el péndulo. Para la latitud de Valdivia, T resulta ser aproximadamente 36 horas, de manera que el plano de oscilación gira unos 10 grados por hora.








\hfill \break
\section{Experimento 8}

\subsection{Contexto Histórico}

\subsection{El experimento}







\hfill \break
\section{Experimento 9}

\subsection{Contexto Histórico}

\subsection{El experimento}








\hfill \break
\section{Experimento 10}

\subsection{Contexto Histórico}

\subsection{El experimento}








%6.6 \times 10^{21} 


% if have a single appendix:
%\appendix[Proof of the Zonklar Equations]
% or
%\appendix  % for no appendix heading
% do not use \section anymore after \appendix, only \section*
% is possibly needed

% use appendices with more than one appendix
% then use \section to start each appendix
% you must declare a \section before using any
% \subsection or using \label (\appendices by itself
% starts a section numbered zero.)
%


%\appendices
%\section{Proof of the First Zonklar Equation}
%Appendix one text goes here.

% you can choose not to have a title for an appendix
% if you want by leaving the argument blank
%\section{}
%Appendix two text goes here.


% use section* for acknowledgment
%\section*{Acknowledgment}


%The authors would like to thank...


% Can use something like this to put references on a page
% by themselves when using endfloat and the captionsoff option.
\ifCLASSOPTIONcaptionsoff
  \newpage
\fi



% trigger a \newpage just before the given reference
% number - used to balance the columns on the last page
% adjust value as needed - may need to be readjusted if
% the document is modified later
%\IEEEtriggeratref{8}
% The "triggered" command can be changed if desired:
%\IEEEtriggercmd{\enlargethispage{-5in}}

% references section

% can use a bibliography generated by BibTeX as a .bbl file
% BibTeX documentation can be easily obtained at:
% http://mirror.ctan.org/biblio/bibtex/contrib/doc/
% The IEEEtran BibTeX style support page is at:
% http://www.michaelshell.org/tex/ieeetran/bibtex/
\bibliographystyle{IEEEtran}
% argument is your BibTeX string definitions and bibliography database(s)
\bibliography{bibliography}



% that's all folks




\end{document}

